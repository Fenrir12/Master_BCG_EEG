%%%%%%%%%%%%%%%%%%%%%%%%%%%%%%%%%%%%%%%%%
% Journal Article
% LaTeX Template
% Version 1.4 (15/5/16)
%
% This template has been downloaded from:
% http://www.LaTeXTemplates.com
%
% Original author:
% Frits Wenneker (http://www.howtotex.com) with extensive modifications by
% Vel (vel@LaTeXTemplates.com)
%
% License:
% CC BY-NC-SA 3.0 (http://creativecommons.org/licenses/by-nc-sa/3.0/)
%
%%%%%%%%%%%%%%%%%%%%%%%%%%%%%%%%%%%%%%%%%

%----------------------------------------------------------------------------------------
%	PACKAGES AND OTHER DOCUMENT CONFIGURATIONS
%----------------------------------------------------------------------------------------

\documentclass[twoside,twocolumn]{article}

\usepackage{blindtext} % Package to generate dummy text throughout this template 

\usepackage[sc]{mathpazo} % Use the Palatino font
\usepackage[T1]{fontenc} % Use 8-bit encoding that has 256 glyphs
\usepackage[utf8]{inputenc}
\linespread{1.05} % Line spacing - Palatino needs more space between lines
\usepackage{microtype} % Slightly tweak font spacing for aesthetics

\usepackage[english]{babel} % Language hyphenation and typographical rules

\usepackage[hmarginratio=1:1,top=32mm,columnsep=20pt]{geometry} % Document margins
\usepackage[hang, small,labelfont=bf,up,textfont=it,up]{caption} % Custom captions under/above floats in tables or figures
\usepackage{booktabs} % Horizontal rules in tables

\usepackage{lettrine} % The lettrine is the first enlarged letter at the beginning of the text

\usepackage{enumitem} % Customized lists
\setlist[itemize]{noitemsep} % Make itemize lists more compact

\usepackage{abstract} % Allows abstract customization
\renewcommand{\abstractnamefont}{\normalfont\bfseries} % Set the "Abstract" text to bold
\renewcommand{\abstracttextfont}{\normalfont\small\itshape} % Set the abstract itself to small italic text

\usepackage[raggedright]{titlesec} % Allows customization of titles
\renewcommand\thesection{\Roman{section}} % Roman numerals for the sections
\renewcommand\thesubsection{\roman{subsection}} % roman numerals for subsections
\titleformat{\section}[block]{\large\scshape\centering}{\thesection.}{1em}{} % Change the look of the section titles
\titleformat{\subsection}[block]{\large}{\thesubsection.}{1em}{} % Change the look of the section titles

\usepackage{fancyhdr} % Headers and footers
\pagestyle{fancy} % All pages have headers and footers
\fancyhead{} % Blank out the default header
\fancyfoot{} % Blank out the default footer
\fancyhead[C]{Real-time DWT processing on mobile platform $\bullet$ Avril 2017 $\bullet$ Projet GTS831} % Custom header text
\fancyfoot[RO,LE]{\thepage} % Custom footer text

\usepackage{titling} % Customizing the title section

\usepackage{hyperref} % For hyperlinks in the PDF
\usepackage{graphicx} 
\graphicspath{ {images/} }

\begin{document}


%----------------------------------------------------------------------------------------
%	ARTICLE CONTENTS
%----------------------------------------------------------------------------------------

\section{Introduction}
%Definition of methods for fast copypaste

%Ballistocardiogram------------------------------------------------------------------------------------------
The Ballistocardiogram (BCG) measures the ballistic forces generated by the heart, that is, the mechanical response of the body when the heart ejects the blood into the vascular tree. This technique has been used since the late 19th century \cite{gordon_certain_1877}, 1877], but failed to justify its usefulness. Still, it has awakened much interest for bridging a gap in basic Electrocardiogram (ECG), that is, a new information about the actual state of the vascular tree and the pumping effect of the heart \cite{singewald_ballistocardiography:_1954}. Lately, with the improvements of signal processing tools and technological advancements, the interest for the BCG was renewed. It is a most useful tool to obtain important information on the cardiac cycles as well as on the respiratory dynamics, while being unobtrusive. Analogously to the ECG, which possesses the so-called QRS complex describing the high-amplitude shape of the electrical stimulations during systolic and dystolic cycles, the BCG possesses the IJK complex. It is then possible to extract information such as the J-J intervals to characterize the cardiac rythm. One of the difficulties in analyzing such signals lie in getting rid of the motion artifacts, that is, when a person moves it body while trying to record the BCG.

%Microbend Fiber Optic Sensor--------------------------------------------------------------------------------
Fiber optic sensors (FOS) are a way of acquiring the mechanical activity of the human body through the modulation of the light in response to an external perturbation during the transmission. They have been thoroughly used due to their high sensitivity and lower cost of production. More precisely, the microbend theory is used for building special types of FOS. Microbending of the optical fiber causes an attenuation of the light transmmited through it, as per critical angle property \cite{}. Multiple pairs of microbenders, when displaced due to physical perturbation, squeezes the fiber. This causes more intensity loss in the cladding region of the optical fiber and increases in function of the pressure exerted on the sensor.

%Wavelets----------------------------------------------------------------------------------------------------
The Direct Wavelet Transform (DWT) is often used for signal denoising as well as for feature extraction. Its strength relies in its ability to split the signal into multiple frequency components. That is, we gain a spectro-temporal representation of this signal. The transform operation actually convolve the signal through  a lowpass and a highpass filter, thus giving detail and smooth coefficients, representative of coarse and finer scale phenomenons. It is possible to go deeper in the decomposition levels by reapplying the passage into the filters to gain detail coefficients at finer scales. Compared to the Fourier transform, the wavelet transform is able to detect discontinuities in the signal without generating too much coefficients to characterize it. 
%------------------------------------------------------------------------------------------------------------

\section{Revue de Littérature}

%Sensors--------------------------------------------------------------------------------
Lots of efforts are directed toward the use of FOS for ballistocardiograms and other biomedical signals. Lau et al. \cite developed a microbend fiber optic sensor able to measure the force exerted on a sensor mat. This extremely sensitive mat was used in an MRI environment, from the perspective of real-time measuring and recording of the breathing rate. The raw data went in a bandpass filter stage and a peak detection algorithm to retrieve the breathing rate of the patient. The novel sensor showed successfull results in acquiring an MRI in a respiratory-gated acquisition, thus preventing respiratory physiological noise from patients in the MRI.

Chen et al. [2012] also used an microbend FOS to extract the BCG from healthy subjects by applying band pass analog and digital filtering to the recorded signal. The BCG is measured on the back of a patient sitting on a chair. They used a NI acquisiton card and a running version of Labview ona computer to acquire the signal and filter it. Analogic filtering was successfully applied directly on the prototype to further reduce more the cost of their system.

Sadek et al. [2015] used a Fiber Bragg Grating Sensor (FBG) to sample the BCG from ten subjects from under a thin bed sheet. 

Dziuda et al. used  an FBG to extract the breathing rate as well as the heart rate from a healthy subject. The sensor is placed between the back of the subject and the back of a chair and undergoes filtering, averaging and referencing with an ECG.

%Signal preprocessing--------------------------------------------------------------------
An important step in retrieving the heart rate is, naturally, the use of a rigorous signal processing to extract the heart beats from the signal. Many sources of noise may pollute the signal, be it the "flicker" noise (1/f noise or pink noise) coming from the electrical equipment, the surrounding power line interference at 50 Hz or 60 Hz depending on the geography or the motion artifact noise. The biggest sources of noise are the motion and breathing artifacts, interfering with the measurements when the subject does even the smallest gesture. To find the BCG in all these sources of noise, many signal processing tools from simple to complex ones are used.

%EMD
Sadek et al. [2015] used an enhanced version of the empirical mode decomposition to extract the BCG from an bragg-grating FOS sensor. This method proved to be reliable. At the ninth decomposition component of the CEEMDAN, they retrieved the heart rate with little error and, together with sensor fusion, they achieved a faster heart rate reading than with EEMD. Pinheiro et al. [2010] also decomposed the BCG time series into few components and found the BCG in motionless recordings and were able to recover part of the heartbeat information. It still was unable to recover most of the information when a motion artifact is involved.

%Wavelets/EMD
With an EMFi capacitive sensor, Pino et al. [2015] compared the EMD and wavelet approach to detect the BCG. They used simple EMD decomposition to extract the BCG. The function modes 2 and 3 contained the heart rate signal but the wavelet approach showed nonetheless better results to get the right results. Their wavelet approach used a Daubechie wavelet base to extract the detail coefficients which contained the bcg signal.

%Wavelets
Sadek et al. [2017] also implemented the Maximal Overlap Direct Wavelet Transform (MODWT) to extract the BCG signal from a microbend FOS. This method proved far more faster than their precedent method using the CEEMDAN algorithm on the data taken on an FBG sensor mat with little more error on the measurements.They did so with a Symlet wavelet base. Combining Donoho and Johnstone [1995] wavelet shrinking method with a another Symlet wavelet base, Jin et al.[2009] detected the heart rate with a peak searching algorithm. Many other teams also used wavelets with different mother wavelets basis to extract the BCG from the raw recordings. Delière et al. [2015] implemented wavelet analysis with a Morlet base to quantify the ballistocardiogram amplitude modulation induced by respiration (BAMR) in an imposed controlled breathing (ICB) protocol. Srinivasan et al. [2015] associated an ECG and an BCG signal through numerical time-frequency modelling employing a Daubechie wavelet base.

%Cepstrum
%----------------------------------------------------------------------------------------
%	REFERENCE LIST
%----------------------------------------------------------------------------------------
\newpage

\bibliographystyle{plain}
\bibliography{biblio}
 

\end{document}
